%%%%%%%%%%%%%%%%%%%%%%%%%%%%%%%%%%%%%%%%%%%%%%%%%%%%%%%%%%%%%%%%%%%%%%
% LaTeX Template: Curriculum Vitae
%
% Source: http://www.howtotex.com/
% Feel free to distribute this template, but please keep the
% referal to HowToTeX.com.
% Date: July 2011
% 
%%%%%%%%%%%%%%%%%%%%%%%%%%%%%%%%%%%%%%%%%%%%%%%%%%%%%%%%%%%%%%%%%%%%%%
% How to use writeLaTeX: 
%
% You edit the source code here on the left, and the preview on the
% right shows you the result within a few seconds.
%
% Bookmark this page and share the URL with your co-authors. They can
% edit at the same time!
%
% You can upload figures, bibliographies, custom classes and
% styles using the files menu.
%
% If you're new to LaTeX, the wikibook is a great place to start:
% http://en.wikibooks.org/wiki/LaTeX
%
%%%%%%%%%%%%%%%%%%%%%%%%%%%%%%%%%%%%%%%%%%%%%%%%%%%%%%%%%%%%%%%%%%%%%%
\documentclass[paper=a4,fontsize=10pt]{scrartcl} % KOMA-article class
\usepackage[top=2.5cm, bottom=2.5cm, left=2.5cm, right=2.5cm]{geometry}
							
\usepackage[english]{babel}
\usepackage[utf8x]{inputenc}
\usepackage{verbatim}
\usepackage[protrusion=true,expansion=true]{microtype}
\usepackage{amsmath,amsfonts,amsthm}     % Math packages
\usepackage{graphicx}                    % Enable pdflatex
\usepackage[dvipsnames]{xcolor}            % Colors by their 'svgnames'
\usepackage{geometry}
	\textheight=700px                    % Saving trees ;-)
%\setlength{\textwidth}{\paperwidth}
%\addtolength{\textwidth}{-1in}
%\calclayout
\usepackage{url}
\usepackage[colorlinks=true,urlcolor=blue,pdftitle={Connor Adams CV},pdfpagemode=FullScreen]{hyperref}
\usepackage{enumitem}
\usepackage{fontawesome5}

\frenchspacing              % Better looking spacings after periods
\pagestyle{empty}           % No pagenumbers/headers/footers

%%% Custom sectioning (sectsty package)
%%% ------------------------------------------------------------
\usepackage{sectsty}

\sectionfont{%			            % Change font of \section command
	\usefont{OT1}{phv}{b}{n}%		% bch-b-n: CharterBT-Bold font
	\sectionrule{0pt}{0pt}{-5pt}{3pt}}

%%% Macros
%%% ------------------------------------------------------------
\newlength{\spacebox}
\settowidth{\spacebox}{8888888888}			% Box to align text
\newcommand{\sepspace}{\vspace*{1em}}		% Vertical space macro

\newcommand{\MyName}[1]{ % Name
		\Huge \usefont{OT1}{phv}{b}{n} \hfill #1
		\par \normalsize \normalfont}
		
\newcommand{\MySlogan}[1]{ % Slogan (optional)
		\large \usefont{OT1}{phv}{m}{n}\hfill \textit{#1}
		\par \normalsize \normalfont}

\newcommand{\NewPart}[1]{\section*{\uppercase{#1}}}

\newcommand{\PersonalEntry}[2]{
		\noindent\hangindent=2em\hangafter=0 % Indentation
		\parbox{\spacebox}{        % Box to align text
		\textit{#1}}		       % Entry name (birth, address, etc.)
		\hspace{1.5em} #2 \par}    % Entry value

\newcommand{\SkillsEntry}[2]{      % Same as \PersonalEntry
		\noindent\hangindent=2em\hangafter=0 % Indentation
		\parbox{\spacebox}{        % Box to align text
		\textit{#1}}			   % Entry name (birth, address, etc.)
		\hspace{1.5em} #2 \par}    % Entry value	
		
\newcommand{\EducationEntry}[4]{
		\noindent \textbf{#1} \hfill      % Study
		\colorbox{Black}{%
			\parbox{5em}{%
			\hfill\color{White}#2}} \par  % Duration
		\noindent \textit{#3} \par        % School
		\noindent\hangindent=2em\hangafter=0 \small #4 % Description
		\normalsize \par}

\newcommand{\WorkEntry}[4]{				  % Same as \EducationEntry
		\noindent \textbf{#1} \hfill      % Jobname
		\colorbox{Black}{\color{White}#2} \par  % Duration
		\noindent \textit{#3} \par              % Company
		\noindent \small #4 % Description
		\normalsize}

%%% Begin Document
%%% ------------------------------------------------------------

\begin{document}
% you can upload a photo and include it here...
%\begin{wrapfigure}{l}{0.5\textwidth}
%	\vspace*{-2em}
%		\includegraphics[width=0.15\textwidth]{photo}
%\end{wrapfigure}

\MyName{{\color{black}Connor} {\color{black}Adams}}
\hfill \PersonalEntry{}{1234 Your Street, My Town, State}
\hfill \PersonalEntry{}{\faIcon{phone-square-alt} (123) 456-7890}
\hfill \PersonalEntry{}{\faIcon{envelope} \href{mailto:email}{email}}
\hfill \PersonalEntry{}{\faIcon{github} \url{https://github.com/Cleeadams}}
\MySlogan{Applied Mathematician}

\sepspace

%%% Personal details
%%% ------------------------------------------------------------
% \NewPart{Personal details}{}

% \PersonalEntry{Phone}{+1 (818) 641-9496}
% \PersonalEntry{GitHub}{\url{https://github.com/Cleeadams}}

%%% Summary
\NewPart{Summary}{}

%%% Option 1
% Determined and enthusiastic instructor with 2 years of experience in a college and high school setting. Educated students who varied in mathematical abilities. A strong sense of developing a learning environment that is engaging as well as enlightening. Designed and researched mathematical models for 2 years at a graduate-level.

%%% Option 2
Dedicated and responsive mathematics instructor with 2 years of experience in classroom management, behavior modification and individualized support. Comfortable working with students of all skill levels to promote learning and boost educational success. Serves as role model by using growth mindset to develop young minds and inspire love of learning. Also, In-depth knowledge of MATLAB, R, Pyton, Microsoft Office, and Latex coupled with communication and problem solving abilities with proven history of mathematical models research at a graduate-level.

%%% Education
%%% ------------------------------------------------------------
\NewPart{Education}{}
%%% Option 1
% \EducationEntry{Cal Poly, Pomona}{2021-2022}{In-Progress: Master's Degree in Applied Mathematics }

% \sepspace

% \EducationEntry{Cal Poly, Pomona}{2016-2020}{Bachelor's Degree in Applied Mathematics}

%%% Option 2

\EducationEntry{MS in Applied Mathematics}{2021 - 2022}{In-Progress: Cal Poly, Pomona}
{
\begin{itemize}
    \item Advisor: Dr. Hubertus von Bremen
    \item Thesis: "Modeling the effects of global warming on Alaskan brown bears"
\end{itemize}
}
\sepspace

\EducationEntry{BA in Applied Mathematics}{2016 - 2020}{Cal Poly, Pomona}


%%% Work experience
%%% ------------------------------------------------------------

\NewPart{Work Experience}{}

\WorkEntry{Graduate Teaching Associate}
{08/2020 - Present}
{Cal Poly, Pomona}
{\begin{itemize}[leftmargin=*]
    \item Lecturer for College Algebra and Trigonometry.
    \item Host activity sections for College Algebra and Introductory Statistics.
    \item Apply collaborative exercises to accelerate the learning of undergraduate students.
    \item establish environments that practice a comfortable and thriving space which allow students the freedom to share and advance ideas in the classroom.
\end{itemize}}

\sepspace

\WorkEntry{Substitute Teacher}
{08/2021 - Present}
{Chino Valley Unified School District}
{\begin{itemize}[leftmargin=*]
    \item Quickly design a lesson and effectively present the information to students.
    \item Adapt to students learning abilities to insure the material is absorbed as well as understood.
    \item Guide and manage students behavior in the classroom.
\end{itemize}}


%%% project
%%% ------------------------------------------------------------
\NewPart{Projects}{}

\WorkEntry{Modeling the effects of global warming on Alaskan brown bears}
{08/2021 - 05/2022}
{Cal Poly, Pomona}
{Supervision: Under Dr. Hubertus von Bremen\\
 Research is in applying logistic and age-structured models to the population of Alaskan brown bears to predict the future of this species as the earth's temperature increases.
}

\sepspace
\newpage
\WorkEntry{Minimum point between two orbits}
{08/2021 - 12/2021}
{Cal Poly, Pomona}
{Utilize steepest decent method to find the minimum distance between two orbits.
This was taken a step further by analysing the hessian which revealed the optimal direction to start steepest decent when the initial staring values reside on an inflection point.
}
\sepspace

\WorkEntry{Solutions of a system with tridiagonal matrix}
{08/2021 - 12/2021}
{Cal Poly, Pomona}
{Discovered the closed form for computing the eigenvalues for a specific type of tridiagonal
matrix. Explored finding the solutions of a system using Gauss-Elimination w/o pivot, Gauss-Elimination w/ partial pivot, and LU factorization to name a few. Results showed that LU factorization and Gauss-Elimination w/o pivot were the best methods because of there efficiency with large tridiagonal matrices.}

\sepspace

\WorkEntry{Coupled systems oscillators}
{01/2021 - 05/2021}
{Cal Poly, Pomona}
{Explored the stability of a single oscillator and coupled system oscillators using MATLAB. Discussed the similarities and differences between the two types of oscillators. Noticed that the coupled system experiences almost
the same stable cycles as the single oscillator.}

\sepspace

\WorkEntry{Logistic equation with delay}
{08/2020 - 12/2020}
{Cal Poly, Pomona}
{Research in the stability of logistic equations with delay. 
Utilized MATLAB to observe solutions and design plots which presented when the delayed-logistic equation becomes unstable. came to the conclusion that the equilibrium
of the population is the carry capacity.}
\sepspace



%%%% Certifications 
%%%% ---------------------------------------------------------
%\NewPart{Certifications}{}

%\WorkEntry{PHP with MySQL}{}{Aspirevision Tech, Microsoft}{}
%\sepspace
%\WorkEntry{Cloud Computing}{}{Electrocloud Labs of NIT Durgapur}{}

%%% Skills
%%% ------------------------------------------------------------
\NewPart{Skills}{}

\SkillsEntry{Languages}{English}
\SkillsEntry{}{Spanish (Entry Level)}
\sepspace

\SkillsEntry{Programming Languages}{ \textsc{Python}, \textsc{R}, \textsc{Matlab}, \textsc{LaTex} }

\sepspace

% \SkillsEntry{Libraries}{\textsc{OpenCV}, \textsc{Pandas}, \textsc{Numpy}, \textsc{Scipy}, \textsc{SMFL}, \textsc{TensorFlow}}

% \sepspace

% \SkillsEntry{Software}{ \textsc{GitHub}, \textsc{Microsoft Office}, \textsc{Anaconda}}
%\sepspace


%\SkillsEntry{Subjects}{\textsc{Compiler Designing}, \textsc{Computer Networks}, \textsc{Algorithms}}
%\SkillsEntry{}{\textsc{Operaing
%System}, \textsc{DBMS}, \textsc{Microprocessor}}
%\SkillsEntry{}{\textsc{Computer Architecture and Organisation }, \textsc{Graphics}}
%\SkillsEntry{}{\textsc{Automata Theory and Natural Languages}}
%\SkillsEntry{}{\textsc{Signals Systems and Circuits}, \textsc{Data Structures}}
%\SkillsEntry{}{\textsc{Discrete Mathematics}, \textsc{Communication Systems}}
%\SkillsEntry{}{\textsc{Digital Logic and Circuit Design}}


%%% References, 
%%% ------------------------------------------------------------
\NewPart{Extra Curricular Activities}{}
%Available upon request
1. Current Science Council Representative of the Society for Industrial \& Applied Mathematics\\
2. Graduate Researcher and Research Presenter\\
3. Teaching Assistant for Statistics\\
\end{document}
